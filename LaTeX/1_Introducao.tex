\chapter{Introdução}
\label{char:intro}
A formação de profissionais na área de Ciência da Computação exige que as ferramentas se aproximem ao máximo da realidade existente. A academia apresenta modelos da realidade que são possíveis de manuseio no ambiente de estudo. A visibilidade que o aprendiz precisa ter sobre o assunto em estudo deve ser semelhante àquela que será encontrada durante o seu exercício profissional. E quando essa similaridade não é encontrada, se torna difícil para o aluno visualizar corretamente em que cenário se aplica o conhecimento adquirido quando este precisar ser utilizado \cite{maia01} \cite{maia03} \cite{kioki08} \cite{silva09}.

As atuais necessidades de aplicação da Ciência da Computação exigem daqueles que desenvolvem sistemas, conhecimento detalhado de como funcionam as ferramentas de trabalho, em especial os métodos e técnicas da Criptografia. Essa necessidade decorre do fato de que a \textit{Internet}, definitivamente, está inserida na vida cotidiana das pessoas e novos costumes estão sendo adotados pela sociedade. O comércio eletrônico, as operações bancárias de transferência eletrônica de fundos, o uso do cartão de crédito como moeda de plástico, por exemplo, se constituem na realidade do convívio social. Junte-se a isso o fato de que a mobilidade na telefonia é uma realidade cristalizada entre as pessoas, independente de qual seja a sua posição social.

Por outro lado, devido à alta complexidade dos algoritmos usados nas aplicações da Criptografia \cite{silva09} \cite{silva12}, a transmissão do conhecimento nem sempre é eficiente o suficiente, principalmente se os algoritmos forem ensinados na sua complexidade real. 

O ensino nas disciplinas que envolvam a temática da segurança da informação, abordando mais especificamente o conteúdo da Criptografia – preocupação prioritária desta pesquisa – acaba sendo prejudicado pela falta de tempo hábil para discorrer sobre o assunto. Naturalmente, a abordagem de cunho pedagógico precisa preservar os aspectos fundamentais da arquitetura de cada algoritmo. \cite{silva09} \cite{garmpis11} \cite{lopes12}

Embora se saiba que, em nome da preservação da eficiência do trabalho acadêmico os algoritmos possam ser apresentados com alguma restrição na sua abrangência, ainda assim, é possível apresentar todo o funcionamento e implementação dos algoritmos durante o período do desenvolvimento de uma disciplina, quer no tempo destinado a uma disciplina de graduação, quer de pós-graduação, com um esforço menor do que aquele necessário para a implementação do algoritmo em sua forma original \cite{maia01} \cite{maia03} \cite{stallings10}.

Analisando o contexto acima apresentado, deriva-se a necessidade de um instrumento que auxilie o processo de ensino nesta área. A finalidade deste trabalho é o desenvolvimento e a implantação uma ferramenta que possibilitará a simulação do funcionamento da versão simplificada do algoritmo criptográfico de chave simétrica conhecido como \acrfull{des}, para utilização de natureza pedagógica.

O produto derivado do desenvolvimento será um simulador destinado a apoiar a formação de profissionais que irão interagir com aplicações que utilizam algoritmos de criptografia por chave simétrica.

Esse simulador permitirá, então, que o algoritmo possa ser executado de modo ininterrupto, ou por etapas discretas, de modo a permitir a completa compreensão do seu funcionamento.

Diante do contexto apresentado, ressalta-se que o propósito desta monografia é contribuir para a resolução da problemática posta, motivo pelo qual define-se como objetivo geral o desenvolvimento de uma ferramenta de simulação de algoritmo criptográfico com intuito de auxiliar o processo de ensino-aprendizagem da técnica de cifra de blocos, nas disciplinas que abordem a criptografia.

Considerando que o desenvolvimento de uma ferramenta de simulação de algoritmos criptográficos é o desafio maior da pesquisa apresentada, é importante ressaltar que historicamente a criptografia por chave secreta, utilizada nessa ferramenta, experimentou um grande impulso por volta do ano de 1974, quando foi apresentado o algoritmo \acrfull{des}. Trata-se de um método para a criptografia de dados baseado em cifra de blocos, que se tornou o padrão usado pelo público e, em particular, pelo governo dos Estados Unidos. Alguns documentos fazem uma distinção entre o \acrshort{des} como um padrão, se referindo ao algoritmo de sua implementação como \acrfull{dea}.

Esse algoritmo herdou os princípios da Cifra de \textit{Feistel}, (1973) resultado de um projeto desenvolvido pela \acrfull{ibm} sobre Criptografia por cifra de blocos. Apesar dos questionamentos sobre a sua vulnerabilidade, por conta do tamanho da chave, o entendimento de como funciona o DES transmite importante conhecimento sobre o mecanismo utilizado nas cifras de bloco.

No ano de 2001, depois de um trabalho de cinco anos, o \acrfull{aes}, conhecido pela sua implementação mais famosa '\textit{Rijndael}', em alusão aos seus criadores, os belgas \textit{Joan Daemen} e \textit{Vincent Rijmen}, passou a ser o novo padrão utilizado para a Criptografia de dados. Trata-se de um método criptográfico também baseado em cifra de bloco e que, tal qual o seu antecessor \acrshort{des}, espera-se da comunidade científica que seja utilizado e detalhadamente analisado.

Dentre as principais características do \textit{Rijndael} se encontra o fato de que o algoritmo ocupa pouca memória, o que o qualifica para ser utilizado em ambientes restritos, tais como telefones celulares e \textit{smart cards}. Justamente por essas características de restrição de necessidades, esse algoritmo também é recomendado pelo \acrfull{ccsds}, organização da qual faz parte o proponente do projeto.

As áreas Empresarial e Acadêmica, que se utilizam de profissionais criados na academia, possuem necessidades criptográficas (segurança nas empresas e didática nas academias) não supridas atualmente devido a falta de uma metodologia eficiente de transmissão do conhecimento para o aluno e pela dificuldade inerente a área de conhecimento citada. \cite{silva09} \cite{garmpis11} \cite{lopes12} \cite{younis20}
