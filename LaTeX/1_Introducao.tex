\chapter{Introdução}
\label{char:intro}
Várias aplicações da Ciência da Computação possuem como necessidade o conhecimento de como funcionam os métodos e técnicas da Criptografia. Essa necessidade decorre do fato de que a \textit{Internet} (principalmente utilizada em dispositivos móveis), definitivamente, está inserida na vida cotidiana das pessoas e novos costumes estão sendo adotados pela sociedade. O comércio eletrônico, as operações bancárias de transferência eletrônica de fundos, o uso do cartão de crédito como moeda de plástico, por exemplo, se constituem na realidade do convívio social, independente de qual seja a sua posição social. Tornando a segurança da informação um ponto cada vez mais importante ao se manipular/trafegar dados.\cite{silva09} \cite{silva12}

A alta complexidade dos algorítimos usados nas aplicações da Criptografia faz com quê o processo de ensino-aprendizagem destes nem sempre seja eficiente, principalmente se os algoritmos forem ensinados e analisados na sua complexidade total \cite{silva09} \cite{silva12}. A academia apresenta aos profissionais em formação na área de Ciência da Computação representações simplificadas do mundo real, embora ainda próximas do mundo real (quanto mais próximas, melhor), com o intuito de facilitar o manuseio dessa representação no ambiente de estudo \cite{maia01} \cite{maia03} \cite{kioki08}. Naturalmente, a abordagem de cunho pedagógico precisa preservar os aspectos fundamentais da arquitetura de cada algoritmo \cite{garmpis11} \cite{lopes12}.

Analisando o contexto acima apresentado, Esse trabalho apresenta uma ferramenta capaz de auxiliar no processo de ensino-aprendizagem de algorítimos criptográficos por cifra de bloco nas disciplinas que abordem a criptografia. A ferramenta possibilita a simulação da execução da versão simplificada do algoritmo criptográfico de chave simétrica conhecido como \acrfull{des}. Esse simulador permite que o algorítimo seja executado passo a passo, tanto de maneira sequencial como cada passo, até mesmo cada etapa dentro de um passo, isoladamente. Cada etapa possui sua explicação e execução bit a bit. Esta é apresentada no capítulo \ref{char:ferrdesenvolvida}.

Além disso, foi realizada uma pesquisa de campo com intuito de avaliar a efetividade da ferramenta no processo de ensino-aprendizagem. Os resultados dessa pesquisa são apresentados no capítulo \ref{char:pesquisa}.
