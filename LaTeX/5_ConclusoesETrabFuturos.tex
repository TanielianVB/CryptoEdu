\chapter{Conclusões e Trabalhos futuros}
\label{char:conclusoesetrabfuturos}
Inicialmente levantou-se a carência de um melhor ensino-aprendizagem da criptografia no meio acadêmico e se explicitou a necessidade desse conhecimento para formando tanto caso ele siga carreira acadêmica quanto caso ele siga carreira empresarial. A alta complexidade dos algorítimos e o tempo reduzido de dedicação para estes na grade curricular podem prejudicar o ensino, mas isso pode ser mitigado com ferramentas voltadas ao ensino criptográfico.

O simulador desenvolvido se destaca em dois pontos sobre os simuladores existentes. Ele é mais acessível e é o primeiro a apresentar explicação e execução passo a passo de todas as etapas envolvidas tanto no processo de criptografia como no processo de descriptografia.

A ferramenta desenvolvida para auxiliar o processo de ensino-aprendizagem da técnica de cifra de blocos nas disciplinas que abordem a criptografia teve sua efetividade comprovada via questionário, onde, após a utilização do simulador, 95,4\% dos entrevistados consideram que o seu nível de conhecimento sobre os processos e etapas apresentados no simulador melhorou e os alunos passaram a ter nível de conhecimento médio sobre esses mesmos processos e etapas quando o seu nível de conhecimento original era nenhum ou pouco.

\section{Oportunidades de melhoria do Simulador}
O simulador foi desenvolvido com foco na reutilização de seus componentes. Preparando-se para incorporar mais facilmente novos algorítimos. A disponibilidade dele no \textit{GitHub} propicia uma maior facilidade de adição de funcionalidades futuras.

Dentre as funcionalidades futuras já prospectadas da ferramenta temos:

\begin{itemize}
    \item Melhorar os pontos escolhidos pelos entrevistados do questionário.
    \item Melhorar a interface quando esta estiver sendo visualizada em \textit{mobile}.
    \item Disponibilizar o simulador em outras línguas além do \textbf{Português-BR} para aumentar a sua abrangência e efetividade.
    \item Adicionar o tema \textbf{escuro}.
    \item Adicionar outros algorítimos ao simulador.
\end{itemize}